\documentclass[11px, a4paper]{article}

\usepackage[utf8]{inputenc}
\usepackage[swedish]{babel}

\usepackage{graphicx}
\usepackage{epstopdf}
\usepackage{color}
\usepackage[final]{pdfpages}
\usepackage{subfigure}
\usepackage{float}
\usepackage{caption}

\usepackage[hyphens]{url}
\usepackage[breaklinks,pdfpagelabels=false]{hyperref}

\usepackage{appendix}
\usepackage{tocvsec2}

\usepackage{biblatex}


% Mail link
\newcommand{\mail}[1]{\href{mailto:#1}{\nolinkurl{#1}}}

\addto{\captionsswedish}{\renewcommand{\abstractname}{}}

 \renewcommand{\familydefault}{\sfdefault}

\bibliography{references}


\hypersetup{
    pdftitle={Picture Hunter},
    pdfauthor={Daniel Ström},
    colorlinks=true,
    citecolor=magenta,
    filecolor=magenta,
    linkcolor=black,
    urlcolor=magenta
}


\title{Picture Hunter}
\author{Daniel Ström \\ \mail{D@nielstrom.se}}

\begin{document}

\maketitle
\begin{abstract}
	\textit{
		Lorem ipsum dolor sit amet, consectetur adipiscing elit. Nam sodales mauris purus, id consequat nisl tempor vel. Donec fermentum, ante at volutpat sodales, arcu magna blandit lectus, vel sollicitudin lorem nisl at enim. Mauris tempus fringilla interdum. Phasellus in tincidunt velit.
	}
\end{abstract}

\tableofcontents

\listoffigures


\section{Introduktion}
	aldskjalkdsj fkla jdklf alksd flk alkdf lkas dfl

\section{Design}

\section{Teknik}

\section{Reflektion}
\subsection{Säkerhet och etik}



\begingroup
\raggedright
\printbibliography
\addcontentsline{toc}{section}{Referenser}
\endgroup

\end{document}