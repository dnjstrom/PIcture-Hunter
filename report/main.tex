\documentclass[11px, a4paper]{article}

\usepackage[utf8]{inputenc}
\usepackage[swedish]{babel}

\usepackage{graphicx}
\usepackage{epstopdf}
\usepackage{color}
\usepackage[final]{pdfpages}
\usepackage{subfigure}
\usepackage{float}
\usepackage{caption}

\usepackage[hyphens]{url}
\usepackage[breaklinks,pdfpagelabels=false]{hyperref}


% Mail link
\newcommand{\mail}[1]{\href{mailto:#1}{\nolinkurl{#1}}}

\addto{\captionsswedish}{\renewcommand{\abstractname}{}}

 \renewcommand{\familydefault}{\sfdefault}


\hypersetup{
    pdftitle={Picture Hunter},
    pdfauthor={Daniel Ström},
    colorlinks=true,
    citecolor=magenta,
    filecolor=magenta,
    linkcolor=black,
    urlcolor=magenta
}


\title{Picture Hunter}
\author{Daniel Ström \\ \mail{D@nielstrom.se}}

\begin{document}

\maketitle
\begin{abstract}
	\textit{
		Lorem ipsum dolor sit amet, consectetur adipiscing elit. Nam sodales mauris purus, id consequat nisl tempor vel. Donec fermentum, ante at volutpat sodales, arcu magna blandit lectus, vel sollicitudin lorem nisl at enim. Mauris tempus fringilla interdum. Phasellus in tincidunt velit.
	}
	\vspace{5mm}
\end{abstract}

\tableofcontents

\listoffigures


\section{Introduktion}
	Picture hunter kan liknas vid en modern smartphone-variant på leken ``Följa John''. Idén är att en spelare tar ett kort på något och sedan skickar kortet till en annan spelare. Den andra spelarens uppgift är att försöka ta samma kort och därigenom ``matcha'' det.

	Ett grundläggande scenario för appens användande har varit att upptäcka en stad. I detta fall ger två personer varandra bilder de tagit någonstanns i staden. För att underlätta detta kan användaren se hur långt bort bilden är tagen (men inte vilken riktning; det är ju ett spel trots allt).

	Gällande kursen är min ambition att appen ska vara bra nog för ett VG.


\section{Design}

	Interaktionen i appen utgörs av tre distinkta aktiviteter: filhantering, bildjämförelse,	bilddelning. Målet för designen av dessa aktiviteter är att skapa en homogen upplevelse som passar in med androids övriga utseende. Dessutom försöker appen vara så transparent som möjligt. Fokus för användaren ska ligga på att försöka hitta bilden, inte på att navigera appen. Som en följd av detta försöker Picture Hunter inte fånga användarens uppmärksammhet genom starka färger i gränssnittet utan använder i den mån det är möjligt standardutseendet för kontroller och ikoner. Då ikoner inte funnits har nya skapats i motsvarande stil.

	De tre aktiviteterna nämnda ovan har delats upp i två olika gränssnitt Det första sköter både filhantering och bilddelning, det andra hanterar bildjämförelse.


\subsection{Filhantering och delning}

	Denna del av appen fungerar i huvudsak som en filhanterare begränsad till appens externa lagringsutrymme. Lagringsutrymmet utgörs inte av appens privata utrymme utan består av en mapp som är nåbara av alla applikationer och som inte tas bort automatiskt vid avinstallering av appen. Detta val är gjort som en följd av designprincipen ``Never lose my stuff''\cite{Principles}.

	Filhanteringen sker i två lager. Det första lagret hanterar ``album'' - sammlingar av bilder. Detta lagret kan ses i figur \ref{fig:albums}. Ett nytt album skapas genom att man klickar på ikonen med en mapp och ett plusstecken. Om man markerar en app genom att hålla ner fingret på den en kort tid visas en ny ActionBar\cite{ContextActionBar} som ger möjligheten att byta namn eller ta bort albumet (se figur \ref{fig:albums_selection}). Detta följer androids designprincip ``Only show what I need when I need it''\cite{Principles}. Omdöpning av album görs med hjälp av en dialogruta anpassad för ändamålet. Vid borttagning av album visas också en dialogruta där användaren måste bekräfta valet\cite{Dialogs}. Dialogrutorna kan ses i figur \ref{fig:albums_selection_rename} och figur \ref{fig:albums_selection_delete}.

	Förutom de olika albumen finns två olika typer av album - egna album och främmande album. Egna album är album med bilder du tagit. Främmande album innehåller bilder du ska försöka matcha. De olika typerna av album presenteras i en horisontellt paginerad lista som rekomenderas för navigering bland kategorier\cite{HorizontalPaging}. För att användaren inte ska tappa bort vart hen befinner sig skrivs den nuvarande positionen ut i det ljusblå bandet ovanför listan med album (uppfyller designprincip ``I should always know where I am''\cite{Principles} ).

	Figur \ref{fig:photos} visar det andra lagret. Detta lager visar bilderna som miniatyrer med deras namn i ett band längst ner (designprincip ``Pictures are faster than words''\cite{Principles}). Även här kan man markera i listan genom att hålla ner fingret. Här ges dock även möjligheten att kopiera eller klippa ut bilderna (se figur \ref{fig:photos_selection}). Ifall något har kopierats visas en klistra-in ikon i den normala ActionBar:en (figur \ref{fig:photos_paste}).

	Ifall användaren håller upp sin telefon mot en annan (som har samma app installerad och vars skärm är upplåst) medans bilder är markerade, skickas dessa till den andra telefonen. Telefonen som tar emot bilderna kommer få en notifikation av Android Beam när filerna är överförda och klickar man på notifikationen öppnas appen automatiskt till ett nytt album som innehåller bilderna.

	Nästan alla ikoner har hämtats från Androids bibliotek för ActionBar ikoner\cite{Icons}. Enda undantaget är ``Nytt Album''-ikonen som är en anpassad version av ``Collections''-ikonen.

	Båda dessa gränssnitt använder horisontell navigering mellan syskon-mappar (dvs. Mina album/Främmande album eller Album 1/Album 2/...). Användaren undviker på detta sätt upp till 50\% av knapptryckningarna som hade behövts om hen behövt backa ett steg för att nå ett syskon. Dessutom uppfyller navigationen designprincipen ``Give me tricks that work everywhere''\cite{Principles}.



\subsection{Bildjämförelse}
	Bildjämförelsen består av två delar. Den första nås när användaren klickar på en omatchad bild och visar en större version av samma bild samt avståndet till vart bilden togs. Denna vy kan ses i figur \ref{fig:detail} Klickar man på bilden snurrar vyn (designprincip ``Delight me in surprising ways''\cite{Principles}) och på baksidan visas istället bilden från telefonens kamera (figur \ref{fig:camera}). Klickar man på kameravyn tas en bild, kameravyn stängs och den tagna bilden jämförs med referensbilden. Huruvida användaren lyckades matcha bilden meddelas med ett Toast-meddelande. Om användaren lyckades visas dessutom en ljusblå kryssruta uppe i högra hörnet. Kryssrutan kan även ses i miniatyrbilden när en bild är korrekt matchad.


\section{Teknik}


\section{Reflektion}

\subsection{Säkerhet och etik}



\begingroup
\raggedright
\bibliographystyle{plain}
\bibliography{references}
\addcontentsline{toc}{section}{Referenser}
\endgroup

\end{document}