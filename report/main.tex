\documentclass[11px, a4paper]{article}

\usepackage[utf8]{inputenc}
\usepackage[swedish]{babel}

\usepackage{graphicx}
\usepackage{epstopdf}
\usepackage{color}
\usepackage[final]{pdfpages}
\usepackage{subfigure}
\usepackage{float}
\usepackage{caption}

\usepackage[hyphens]{url}
\usepackage[breaklinks,pdfpagelabels=false]{hyperref}


% Mail link
\newcommand{\mail}[1]{\href{mailto:#1}{\nolinkurl{#1}}}

\addto{\captionsswedish}{\renewcommand{\abstractname}{}}

 \renewcommand{\familydefault}{\sfdefault}


\hypersetup{
    pdftitle={Picture Hunter},
    pdfauthor={Daniel Ström},
    colorlinks=true,
    citecolor=magenta,
    filecolor=magenta,
    linkcolor=black,
    urlcolor=magenta
}


\title{Picture Hunter}
\author{Daniel Ström \\ \mail{D@nielstrom.se}}

\begin{document}

\maketitle
\begin{abstract}
	\textit{
		Lorem ipsum dolor sit amet, consectetur adipiscing elit. Nam sodales mauris purus, id consequat nisl tempor vel. Donec fermentum, ante at volutpat sodales, arcu magna blandit lectus, vel sollicitudin lorem nisl at enim. Mauris tempus fringilla interdum. Phasellus in tincidunt velit.
	}
	\vspace{5mm}
\end{abstract}

\tableofcontents

\listoffigures


\section{Introduktion}
	Picture hunter kan liknas vid en modern smartphone-variant på leken ``Följa John''. Idén är att en spelare tar ett kort på något och sedan skickar kortet till en annan spelare. Den andra spelarens uppgift är att försöka ta samma kort och därigenom ``matcha'' det.

	Ett grundläggande scenario för appens användande har varit att upptäcka en stad. I detta fall ger två personer varandra bilder de tagit någonstanns i staden. För att underlätta detta kan användaren se hur långt bort bilden är tagen (men inte vilken riktning; det är ju ett spel trots allt).

	Gällande kursen är min ambition att appen ska vara bra nog för ett VG.


\section{Design}

	Interaktionen i appen utgörs av tre distinkta aktiviteter: filhantering, bildjämförelse,	bilddelning. Målet för designen av dessa aktiviteter är att skapa en homogen upplevelse som passar in med androids övriga utseende. Dessutom försöker appen vara så transparent som möjligt. Fokus för användaren ska ligga på att försöka hitta bilden, inte på att navigera appen. Som en följd av detta försöker Picture Hunter inte fånga användarens uppmärksammhet genom starka färger i gränssnittet utan använder i den mån det är möjligt standardutseendet för kontroller och ikoner. Då ikoner inte funnits har nya skapats i motsvarande stil.

	De tre aktiviteterna nämnda ovan har delats upp i två olika gränssnitt Det första sköter både filhantering och bilddelning, det andra hanterar bildjämförelse.


\subsection{Filhantering och delning}

	Denna del av appen fungerar i huvudsak som en filhanterare begränsad till appens externa lagringsutrymme. Lagringsutrymmet utgörs inte av appens privata utrymme utan består av en mapp som är nåbara av alla applikationer och som inte tas bort automatiskt vid avinstallering av appen. Detta val är gjort som en följd av designprincipen ``Never lose my stuff''\cite{Principles}.

	Filhanteringen sker i två lager. Det första lagret hanterar ``album'' - sammlingar av bilder. Detta lagret kan ses i figur \ref{fig:albums}. Ett nytt album skapas genom att man klickar på ikonen med en mapp och ett plusstecken. Om man markerar en app genom att hålla ner fingret på den en kort tid visas en ny ActionBar\cite{ContextActionBar} som ger möjligheten att byta namn eller ta bort albumet (se figur \ref{fig:albums_selection}). Detta följer androids designprincip ``Only show what I need when I need it''\cite{Principles}. Omdöpning av album görs med hjälp av en dialogruta anpassad för ändamålet. Vid borttagning av album visas också en dialogruta där användaren måste bekräfta valet\cite{Dialogs}. Dialogrutorna kan ses i figur \ref{fig:albums_selection_rename} och figur \ref{fig:albums_selection_delete}.

	Förutom de olika albumen finns två olika typer av album - egna album och främmande album. Egna album är album med bilder du tagit. Främmande album innehåller bilder du ska försöka matcha. De olika typerna av album presenteras i en horisontellt paginerad lista som rekomenderas för navigering bland kategorier\cite{HorizontalPaging}. För att användaren inte ska tappa bort vart hen befinner sig skrivs den nuvarande positionen ut i det ljusblå bandet ovanför listan med album (uppfyller designprincip ``I should always know where I am''\cite{Principles} ).

	Figur \ref{fig:photos} visar det andra lagret. Detta lager visar bilderna som miniatyrer med deras namn i ett band längst ner (designprincip ``Pictures are faster than words''\cite{Principles}). Även här kan man markera i listan genom att hålla ner fingret. Här ges dock även möjligheten att kopiera eller klippa ut bilderna (se figur \ref{fig:photos_selection}). Ifall något har kopierats visas en klistra-in ikon i den normala ActionBar:en (figur \ref{fig:photos_paste}).

	Ifall användaren håller upp sin telefon mot en annan (som har samma app installerad och vars skärm är upplåst) medans bilder är markerade, skickas dessa till den andra telefonen. Telefonen som tar emot bilderna kommer få en notifikation av Android Beam när filerna är överförda och klickar man på notifikationen öppnas appen automatiskt till ett nytt album som innehåller bilderna.

	Nästan alla ikoner har hämtats från Androids bibliotek för ActionBar ikoner\cite{Icons}. Enda undantaget är ``Nytt Album''-ikonen som är en anpassad version av ``Collections''-ikonen.

	Båda dessa gränssnitt använder horisontell navigering mellan syskon-mappar (dvs. Mina album/Främmande album eller Album 1/Album 2/...). Användaren undviker på detta sätt upp till 50\% av knapptryckningarna som hade behövts om hen behövt backa ett steg för att nå ett syskon. Dessutom uppfyller navigationen designprincipen ``Give me tricks that work everywhere''\cite{Principles}.



\subsection{Bildjämförelse}
	Bildjämförelsen består av två delar. Den första nås när användaren klickar på en omatchad bild och visar en större version av samma bild samt avståndet till vart bilden togs. Denna vy kan ses i figur \ref{fig:detail} Klickar man på bilden snurrar vyn (designprincip ``Delight me in surprising ways''\cite{Principles}) och på baksidan visas istället bilden från telefonens kamera (figur \ref{fig:camera}). Klickar man på kameravyn tas en bild, kameravyn stängs och den tagna bilden jämförs med referensbilden. Huruvida användaren lyckades matcha bilden meddelas med ett Toast-meddelande. Om användaren lyckades visas dessutom en ljusblå kryssruta uppe i högra hörnet. Kryssrutan kan även ses i miniatyrbilden när en bild är korrekt matchad.


\section{Teknik}

	Appen är uppbyggd med tre aktiviteter - en för album-vyn, en för bild-vyn och en för jämförelse. De första två hade nog kunnat slås ihop till en ända aktivitet, men uppdelningen hjälper till att separera vad som ska hända vid användarinteraktion. Genom att gemensam kod extraherats till egna klasser dupliceras dessutom kod endast marginellt.

	Samtliga gränssnitt implementeras i huvudsak som fragment som i största möjliga mån är fristående från dess aktivitet. Vid behov används dock aktiviten som en mellanhand mellan fragmenten\cite{FragmentCommunication}. Ett tydligt exempel på detta är ComparisonActivity som startar och avslutar de båda fragmenten efter vad som är lämpligt och som tar hand om bilden som kameran tagit.

	Appen uppvisar flera tekniska finesser. Några av dessa är ganska nyligen tillgada i Android så applikationen kräver en tämligen uppdaterad telefon. Den kräver även att telefonen har ett SD-kort, en kamera och NFC-förmåga.

\subsection{Bildigenkänning}
\label{subsec:image_recog}
	Den mest grundläggande funktionaliteten i appen är jämörelsen av två bilder. I början fanns förhoppningen att denna funkion skulle finnas i Androids api eller som ett externt bibliotek till java. Ingen av dessa möjligheter visade sig tyvärr möjliga varpå jag försökte mig på att implementera jämförelsen själv med hjälp av en post på \emph{Java Image Processing Cookbook}\cite{ImageComparison}.

	Jämförelsen görs genom att algoritmen väljer ut 100 punkter på varje bild och räknar ut den genomsnittsliga färgen i punktens omgivningen. Varje par punkter (en från varje bild) med samma position jämförs sedan var för sig och avståndet mellan färgerna i de två punkterna adderas till en total. Avstådet mellan två färger definieras i det här fallet som avstådet i ett tredimensionellt rum där röd, grön och blå utgör axlarna och med svart i origo. När det totala avståndet mellan bilderna räknats ut kan man jämföra detta med ett tröskelvärde som avgör hur mycket ``fel'' man tillåts. Enkelt uttryckt kan man alltså säga att algoritmen komprimerar bilden till 100x100 pixlar och sedan jämför färgen på bilderna för varje pixel.

	Det finns såklart flera problem med att bara jämföra färgvärden. Till exempel genererar endast stora förändringar i form ett betydande avstånd och små detaljer försvinner in i bakgrunden när bilden ``komprimeras''. Jämförelsen är också mycket känslig för olika ljusförhållanden. Att jämförelsen inte är exakt bör dock ses som en fördel då en viss grad av fel bör tillåtas för att jämförelsen ska vara praktiskt applicerbar. Skulle en liknande app byggas för kommersiellt bruk behöver man dock givetvis använda en mer avancerad algoritm.

\subsection{Kamera}
\label{subsec:camera}
	Då olika telefoner kan ta bilder i olika upplösning behövs någon standard införas på bilderna för att de ska kunna jämföras på ett rättvist sätt av algoritmen i avsnitt \ref{subsec:image_recog}. I vanliga fall skulle uppgiften att ta en bild delegeras till telefonens standardkameraapp men ett Intent med ACTION\_IMAGE\_CAPTURE har ingen möjlighet att påverka den slutliga storleken. Man hade kunnat normalisera bilden efteråt men då tar användaren en avlång bild men en liksidig sparas till disk. För att inte överraska användaren behöver kamerans bild motsvara slutresultatet. För att åstadkomma detta implementerades ett specialgjort kamerafragment. Kamerafragmentet används till all bildtagning i appen och begränsar bilderna till 1024x1024 pixlar.

\subsection{GPS}
	När en kameran i avsnitt \ref{subsec:camera} öppnas börjar appen fråga efter användarens koordinater. Om dessa funnits i tid till att bilden tas sparas dessa tillsammans med bilden som metadata och används för att ge spelaren som ska matcha bilden en idé om avståndet till platsen där bilden tagits.

\subsection{Near Field Communication (NFC)}
	För att dela bilder mellan två telefoner använder sig appen av Near Field Communication i from av Android Beam. Överföring av filer med hjälp av Android Beam lades till först i api-nivå 16 vilket sätter detta som minimi-sdk-krav för att kunna använda appen. Så fort telefonen förs nära en annan (ca 4cm) kommer appen kontrollera om det finns bilder markerade. Dessa förs i så fall över till den andra telefonen och appen öppnas i så fall via ett speciellt Intent Filter. På grund av hur Android Beam är implementerad levereras de nya filerna med hjälp av en Content Provider som appen läser från och flyttar in filerna till appens filarea.

\subsection{ViewPager}
	ViewPager är en del av a support library.

\subsection{Animation}

\subsection{Specialgjorda vyer och attribut}

\subsection{Exif metadata}

\subsection{Backgrundsarbete och laddning}


\section{Reflektion}

\subsection{Säkerhet och etik}



\begingroup
\raggedright
\bibliographystyle{plain}
\bibliography{references}
\addcontentsline{toc}{section}{Referenser}
\endgroup

\end{document}